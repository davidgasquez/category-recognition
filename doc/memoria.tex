\documentclass[a4paper,12pt]{article}

%----------------------------------------------------------------------------------------
% Paquetes
%----------------------------------------------------------------------------------------

% Permite reconocer acentos
\usepackage[utf8]{inputenc}
% Salida correcta para copiar desde el PDF
\usepackage[T1]{fontenc}
% Anchura correcta de los márgenes
\usepackage{a4wide}
% Usar la fuente Latin Modern
\usepackage{lmodern}
% Establece el idioma del documento
\usepackage[spanish]{babel}
% Permite instertar imágenes
\usepackage{graphicx}
% Control de los párrafos
\usepackage{parskip}
% Modifica el índice
\usepackage[nottoc,numbib]{tocbibind}
% Diferentes tamaños de letra
\usepackage{anyfontsize}
% AMS
\usepackage{amsmath, amsfonts, amssymb}
% Mejor apariencia del texto
\usepackage{microtype}
% Personalizar los encabezados y pies de página
\usepackage{fancyhdr}
% Mejores tablas
\usepackage{booktabs}
% Hipervínculos
\usepackage[colorlinks=false, pdfborder={0 0 0}]{hyperref}

% Información del Documento
\title{
\huge \bfseries Memoria
}
\author{David Gasquez}
\date{}

% Indentar los párrafos
\setlength{\parindent}{\baselineskip}
\linespread{1.25}

%----------------------------------------------------------------------------------------
% Inicio del Documento
%----------------------------------------------------------------------------------------

\begin{document}

% Incluir la portada
\begin{titlepage}

% Comando para dibujar las líneas horizontales
\newcommand{\HRule}{\rule{\linewidth}{0.7mm}}

% Centrar todo
\center 
 
%----------------------------------------------------------------------------------------
% Sub-títulos
%----------------------------------------------------------------------------------------

\textsc{\Huge Visión Por Computador}\\[2.5cm] 
\textsc{\LARGE Proyecto Final}\\[3cm]


%----------------------------------------------------------------------------------------
% Título
%----------------------------------------------------------------------------------------

\HRule \\[1cm]
{\fontsize{55}{60}\selectfont \sffamily Memoria}\\[0.6cm] 
\HRule \\[8.2cm]
 

%----------------------------------------------------------------------------------------
% Información adicional
%----------------------------------------------------------------------------------------

\begin{minipage}{0.4\textwidth}
\begin{flushleft} \large
David \textsc{Gasquez}\\ % Your name
\texttt{76421093M}\\
davidgasquez@gmail.com
\end{flushleft}
\end{minipage}
~
\begin{minipage}{0.4\textwidth}
\begin{flushright} \large
\end{flushright}
\end{minipage}\\[4cm]


\end{titlepage}



%----------------------------------------------------------------------------------------
% Índice
%----------------------------------------------------------------------------------------
\pagestyle{empty}
\renewcommand{\contentsname}{\centering Índice}
\tableofcontents
\newpage


%----------------------------------------------------------------------------------------
% Estilo de las cabeceras y de los piés de página
%----------------------------------------------------------------------------------------

\pagestyle{fancy}
\cfoot{\thepage}
\rhead{}
\renewcommand{\headrulewidth}{0pt}
\renewcommand{\footrulewidth}{0.4pt}
\renewcommand{\headheight}{15pt}


%----------------------------------------------------------------------------------------
% Contenido
%----------------------------------------------------------------------------------------

\section{Descripción del Problema}

\newpage
\section{Implementación}

\newpage
\section{Experimentación}

\newpage
\section{Resultados}

\newpage
\section{Conclusiones}

%----------------------------------------------------------------------------------------
% Bibliografía
%----------------------------------------------------------------------------------------

\newpage
\begin{thebibliography}{99}

\end{thebibliography}



\end{document}
