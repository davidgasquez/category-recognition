\documentclass[a4paper,12pt]{article}

%-------------------------------------------------------------------------------
% Paquetes
%-------------------------------------------------------------------------------

% Permite reconocer acentos
\usepackage[utf8]{inputenc}
% Salida correcta para copiar desde el PDF
\usepackage[T1]{fontenc}
% Anchura correcta de los márgenes
\usepackage{a4wide}
% Usar la fuente Latin Modern
\usepackage{lmodern}
% Establece el idioma del documento
\usepackage[spanish]{babel}
% Permite instertar imágenes
\usepackage{graphicx}
% Control de los párrafos
\usepackage{parskip}
% Modifica el índice
\usepackage[nottoc,numbib]{tocbibind}
% Diferentes tamaños de letra
\usepackage{anyfontsize}
% AMS
\usepackage{amsmath, amsfonts, amssymb}
% Mejor apariencia del texto
\usepackage{microtype}
% Personalizar los encabezados y pies de página
\usepackage{fancyhdr}
% Mejores tablas
\usepackage{booktabs}
% Hipervínculos
\usepackage[colorlinks=false, pdfborder={0 0 0}]{hyperref}

% Información del Documento
\title{
\huge \bfseries Memoria
}
\author{David Gasquez}
\date{}

% Indentar los párrafos
\setlength{\parindent}{\baselineskip}
\linespread{1.25}

%-------------------------------------------------------------------------------
% Inicio del Documento
%-------------------------------------------------------------------------------

\begin{document}

% Incluir la portada
\begin{titlepage}

% Comando para dibujar las líneas horizontales
\newcommand{\HRule}{\rule{\linewidth}{0.7mm}}

% Centrar todo
\center 
 
%----------------------------------------------------------------------------------------
% Sub-títulos
%----------------------------------------------------------------------------------------

\textsc{\Huge Visión Por Computador}\\[2.5cm] 
\textsc{\LARGE Proyecto Final}\\[3cm]


%----------------------------------------------------------------------------------------
% Título
%----------------------------------------------------------------------------------------

\HRule \\[1cm]
{\fontsize{55}{60}\selectfont \sffamily Memoria}\\[0.6cm] 
\HRule \\[8.2cm]
 

%----------------------------------------------------------------------------------------
% Información adicional
%----------------------------------------------------------------------------------------

\begin{minipage}{0.4\textwidth}
\begin{flushleft} \large
David \textsc{Gasquez}\\ % Your name
\texttt{76421093M}\\
davidgasquez@gmail.com
\end{flushleft}
\end{minipage}
~
\begin{minipage}{0.4\textwidth}
\begin{flushright} \large
\end{flushright}
\end{minipage}\\[4cm]


\end{titlepage}



%-------------------------------------------------------------------------------
% Índice
%-------------------------------------------------------------------------------
\pagestyle{empty}
\renewcommand{\contentsname}{\centering Índice}
\tableofcontents
\newpage


%-------------------------------------------------------------------------------
% Estilo de las cabeceras y de los piés de página
%-------------------------------------------------------------------------------

\pagestyle{fancy}
\cfoot{\thepage}
\rhead{}
\renewcommand{\headrulewidth}{0pt}
\renewcommand{\footrulewidth}{0.4pt}
\renewcommand{\headheight}{15pt}


%-------------------------------------------------------------------------------
% Contenido
%-------------------------------------------------------------------------------

\section{Descripción del Problema}

\subsection{Introducción}

El problema al que nos enfrentamos se trata de la clasificación automática de
la categoría a la que pertenece un objeto o varios. Es uno de los temas más 
activos en investigación actualmente. Actualmente el reconocimiento de la
instancia de un objeto en particular está bastante asentado. El problema
pués, se trata de reconocer la categoría a la que pertenece un objeto en
cuestión y no de que objeto específico se trata. La correcta categorización 
de un objeto nos proporciona información adicional sobre el mismo, como puede
ser su uso.

El enfoque de esté trabajo va a ser la clasificación de objetos en 101 
distintas categorías usando la base de dátos pública del
California Institute of Technology.

La clasificación en nuestro caso será indicar la presencia o nó de una categoría
en una imagen dada. Asignar una categoría a una imagen presenta varios 
problemas:
\begin{itemize}
  \item Diferencias en el aspecto del objeto
  \item La imagen puede tomarse en distintas condiciones,posiciones\ldots
  \item Puede ser subjetivo, dependiendo del observador
\end{itemize}

\subsection{Procedimiento}
\begin{enumerate}
  \item Extraer un número considerable de descriptores SIFT(entre 50.000 y 
  100.000) de un conjunto de imágenes cualesquiera
  \item Realizar entre 200 y 1500 clústeres usándo la técnica KNN con los
  descriptores SIFT
  \item Obtener de cada categoría un conjunto de imágenes de entrenamiento, de
  las cuales obtenemos los descriptores SIFT densos para contrastarlos con los
  clústeres y crear el descriptor de la imagen
  \item Entrenar una máquina de soporte vectorial con dichos descriptores para
  cada clase
  \item Comprobar los resultados con un conjunto de test
\end{enumerate}

\newpage
\section{Implementación}

\newpage
\section{Experimentación}

\newpage
\section{Resultados}

\newpage
\section{Conclusiones}

%-------------------------------------------------------------------------------
% Bibliografía
%-------------------------------------------------------------------------------

\newpage
\begin{thebibliography}{99}

\end{thebibliography}



\end{document}
